\begin{abstract}
We establish fundamental complexity bounds for automated theorem discovery in geometric reasoning systems. Our main result proves that for any geometric theorem discovery system $G$ operating in first-order logic with equality, there exists a complexity threshold $T(n) = O(n^3 \log n)$ such that allocating computational resources exceeding $T(n)$ yields probability $1 - e^{-n}$ of discovering a novel theorem of statement length $n$ as $n \to \infty$. The proof introduces a measure-theoretic framework for analyzing the space of geometric statements and shows that heuristic search trajectories with algebraic verification form martingale sequences. By applying the Azuma-Hoeffding inequality and the method of moments, we demonstrate convergence to the claimed asymptotic behavior. We further prove this threshold is tight through explicit construction of adversarial statement spaces. Our results provide the first rigorous complexity bounds for automated geometric theorem discovery and establish fundamental limits on the computational resources required for reliable theorem generation. The techniques developed here extend to broader classes of formal reasoning systems and suggest a general framework for analyzing the complexity of automated mathematical discovery.
\end{abstract}