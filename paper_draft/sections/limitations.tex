\section{Limitations and Future Work}

While our complexity analysis establishes important theoretical bounds for geometric theorem discovery, several key limitations must be acknowledged:

\textbf{Scope Limitation:} Our framework applies exclusively to first-order logic with equality, which restricts the expressiveness of geometric reasoning. Specifically, statements requiring higher-order logic or modal operators for concepts like continuity are not captured by our $O(n^3 \log n)$ bound. The theorem space $\mathcal{S}_n$ we consider excludes important geometric concepts that require richer logical frameworks.

\textbf{Computational Complexity Assumptions:} The proof of Lemma 1, which bounds the cardinality of the search space by $2^{n \log n}$, assumes that verification of geometric statements can be performed within polynomial time. This assumption may not hold for statements involving:

\begin{itemize}
\item Transcendental numbers or infinite processes
\item Geometric constructions requiring arbitrary precision
\item Statements where verification itself is undecidable
\end{itemize}

\textbf{Measure-Theoretic Constraints:} Our analysis requires a well-defined measure $\mu$ over the space of geometric statements $\mathcal{S}_n$. For statements involving continuous parameters, this measure may become ill-defined. While we prove that:

\[
\mu(\mathcal{S}_n) \leq 2^{n \log n}
\]

this bound assumes effective discretization of the statement space, which may not be valid for all geometric theorems.

\textbf{Resource Limitations:} Though asymptotically polynomial, the complexity bound $T(n) = O(n^3 \log n)$ can still be prohibitive for practical applications. For a modest geometric statement with $n = 100$ logical symbols, the computational resources required may exceed practical limits, despite being theoretically tractable.

Future work should address these limitations by:
1. Extending the framework to higher-order geometric reasoning systems
2. Developing tighter bounds for restricted classes of geometric statements
3. Investigating alternative measure spaces that better capture geometric complexity
4. Exploring practical approximation algorithms that trade theoretical completeness for computational efficiency

These extensions would significantly enhance the practical applicability of our theoretical results to real-world geometric theorem discovery systems.