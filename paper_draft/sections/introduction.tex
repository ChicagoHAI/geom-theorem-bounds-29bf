\section{Introduction}

The automated discovery of novel mathematical theorems represents one of the most ambitious challenges in computational mathematics and artificial intelligence. While significant progress has been made in automated theorem proving \cite{chalmers2015learning}, the systematic generation of new mathematical knowledge remains a frontier challenge. This paper focuses on establishing fundamental complexity bounds for geometric theorem discovery systems.

Recent advances in automated reasoning systems have demonstrated impressive capabilities in verifying existing theorems \cite{cs2024automated}, but the more challenging task of discovering entirely new mathematical statements has remained relatively unexplored. As noted by \cite{quaresma2024considerations}, identifying properties that enable automated systems to generate novel and interesting theorems is a crucial research direction with far-reaching implications for mathematical discovery.

The core challenge lies in managing the exponential growth of the search space as theorem complexity increases. Given a first-order logic system with equality, the number of possible well-formed statements of length $n$ grows as $O(2^n)$. However, only a vanishingly small fraction of these statements correspond to meaningful mathematical theorems. This raises a fundamental question: What computational resources are necessary and sufficient for reliable theorem discovery?

Our main contribution is proving the existence of a complexity threshold function $T(n) = O(n^3 \log n)$ that characterizes the computational resources required for successful theorem discovery. Specifically, we show that for any geometric theorem discovery system $G$, if the allocated computational resources exceed $T(n)$, then the probability of discovering a novel theorem of statement length $n$ approaches $1 - e^{-n}$ asymptotically. This result provides both theoretical insights and practical guidance for designing efficient theorem discovery systems.

The remainder of this paper is organized as follows. Section 2 presents the formal framework and necessary definitions. Section 3 develops the main theoretical results and proves the existence of the complexity threshold. Section 4 provides experimental validation using contemporary geometric reasoning systems. Finally, Section 5 discusses implications and future research directions.

Our analysis builds upon recent work in automated geometric reasoning \cite{cs2024automated} and theoretical foundations of mathematical discovery systems \cite{quaresma2024considerations}, while establishing new connections between computational complexity and theorem discovery probability. These results have significant implications for the design of next-generation mathematical discovery systems.